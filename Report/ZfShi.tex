%% This is file `elsarticle-template-1-num.tex',
%%
%% Copyright 2009 Elsevier Ltd
%%
%% This file is part of the 'Elsarticle Bundle'.
%% ---------------------------------------------
%%
%% It may be distributed under the conditions of the LaTeX Project Public
%% License, either version 1.2 of this license or (at your option) any
%% later version.  The latest version of this license is in
%%    http://www.latex-project.org/lppl.txt
%% and version 1.2 or later is part of all distributions of LaTeX
%% version 1999/12/01 or later.
%%
%% Template article for Elsevier's document class `elsarticle'
%% with numbered style bibliographic references
%%
%% $Id: elsarticle-template-1-num.tex 149 2009-10-08 05:01:15Z rishi $
%% $URL: http://lenova.river-valley.com/svn/elsbst/trunk/elsarticle-template-1-num.tex $
%%

% \documentclass[preprint,12pt]{elsarticle}


%% Use the option review to obtain double line spacing
%% \documentclass[preprint,review,12pt]{elsarticle}

%% Use the options 1p,twocolumn; 3p; 3p,twocolumn; 5p; or 5p,twocolumn
%% for a journal layout:

\documentclass[final,1p,times]{elsarticle}
\usepackage{geometry}
\geometry{left=2cm, right=2cm, top=1.5cm, bottom=1.5cm}
%% \documentclass[final,1p,times,twocolumn]{elsarticle}
%% \documentclass[final,3p,times]{elsarticle}
%% \documentclass[final,3p,times,twocolumn]{elsarticle}
%% \documentclass[final,5p,times]{elsarticle}
%% \documentclass[final,5p,times,twocolumn]{elsarticle}

%% The graphicx package provides the includegraphics command.
\usepackage{graphicx}
%% The amssymb package provides various useful mathematical symbols
\usepackage{amssymb}
%% The amsthm package provides extended theorem environments
%% \usepackage{amsthm}

%% The lineno packages adds line numbers. Start line numbering with
%% \begin{linenumbers}, end it with \end{linenumbers}. Or switch it on
%% for the whole article with \linenumbers after \end{frontmatter}.
\usepackage{lineno}

\usepackage{xcolor,listings}
\usepackage[colorlinks,linkcolor=black,urlcolor=black]{hyperref}

%% natbib.sty is loaded by default. However, natbib options can be
%% provided with \biboptions{...} command. Following options are
%% valid:

%%   round  -  round parentheses are used (default)
%%   square -  square brackets are used   [option]
%%   curly  -  curly braces are used      {option}
%%   angle  -  angle brackets are used    <option>
%%   semicolon  -  multiple citations separated by semi-colon
%%   colon  - same as semicolon, an earlier confusion
%%   comma  -  separated by comma
%%   numbers-  selects numerical citations
%%   super  -  numerical citations as superscripts
%%   sort   -  sorts multiple citations according to order in ref. list
%%   sort&compress   -  like sort, but also compresses numerical citations
%%   compress - compresses without sorting
%%
%% \biboptions{comma,round}

% \biboptions{}

\journal{Database System Technology}

\begin{document}

\begin{frontmatter}

%% Title, authors and addresses

\title{Lab Project: OpenStreetMap}

%% use the tnoteref command within \title for footnotes;
%% use the tnotetext command for the associated footnote;
%% use the fnref command within \author or \address for footnotes;
%% use the fntext command for the associated footnote;
%% use the corref command within \author for corresponding author footnotes;
%% use the cortext command for the associated footnote;
%% use the ead command for the email address,
%% and the form \ead[url] for the home page:
%%
%% \title{Title\tnoteref{label1}}
%% \tnotetext[label1]{}
%% \author{Name\corref{cor1}\fnref{label2}}
%% \ead{email address}
%% \ead[url]{home page}
%% \fntext[label2]{}
%% \cortext[cor1]{}
%% \address{Address\fnref{label3}}
%% \fntext[label3]{}


%% use optional labels to link authors explicitly to addresses:
%% \author[label1,label2]{<author name>}
%% \address[label1]{<address>}
%% \address[label2]{<address>}

\author{Zhenfeng Shi}
\author{Hongru Zhu}
\author{Chang Zhou}

\address{5130309777, 5130309784, 5130309787}

\begin{abstract}
OpenStreetMap powers map data on thousands of web sites, mobile apps, and hardware devices. It is built by a community of mappers that contribute and maintain data about roads, trails, cafés, railway stations, and much more, all over the world. Our task is to load the OpenStreetMap data into MySQL database system, and make suitable tables, indexes and queries for them.
\end{abstract}

\begin{keyword}
OSM \sep Database
%% keywords here, in the form: keyword \sep keyword

%% MSC codes here, in the form: \MSC code \sep code
%% or \MSC[2008] code \sep code (2000 is the default)

\end{keyword}

\end{frontmatter}

%%
%% Start line numbering here if you want
%%
\linenumbers

%% main text
\section{Usage}
\subsection{Environment}
MySQL + Python 3 + pymysql + Firefox + Django + leaflet
\subsection{File Structure}
The following tree demonstrate the useful files in the folders, other files can be ignored.
\begin{figure}[thpb]
      \centering
      \includegraphics[width=6cm]{tree.png}
      \caption{Tree for files}
      \label{fig:tree}
\end{figure}
\subsection{Install}
Enter the root path of this project, run the following command in the shell:
\begin{verbatim}
python SZZ_install.py [-h] [-c host] [-u user] [-p passwd] [-n dbname] [-i input]

                     -c:  host connect, for instance 'localhost'
                     -u:  username for mysql, for instance 'root'
                     -p:  password for mysql, ignore this if no password
                     -n:  name for the new database
                     -i:  inputfile path, for instance '../shanghai_dump.osm'
\end{verbatim}
For instance,

\begin{verbatim}
python SZZ_install -c localhost -u root -n OSM -i data/shanghai_dump.osm
\end{verbatim}

\subsection{Queries}
Website Overview\\

A part: logo part. Our website is called Espressionmilk having a wish that your experience in using our website could be as silky as sipping a cup of Espressionmilk.\\

B part: query type select \& search input. Here you can select query type and input content\\

C part: result output in detail. It contains all the useful information in a clearer view.\\

D part: result output on map. It gives reader an intuitive view of the result on the map.

\begin{figure}[thpb]
      \centering
      \includegraphics[width=14cm]{weboverivew.png}
      \caption{Website Overview}
      \label{fig:WO}
\end{figure}

You need to install Django at first. Before using the website, you have to start the server:
\begin{verbatim}
              python manage.py runserver
\end{verbatim}

Then enter the address in your browser: http://localhost:8000/query-post.

You can use select menu to choose the query type.\\
\fbox{%  
  \parbox{\textwidth}{%  
      Query 1\\
      input format: nodeID\\
      You can get a sentence indicate whether it is an intersection and a table of the wayID that contains this node.

      Query 2\\
      input format: wayID\\
      You can get nodeID of the nodes along this way

      Query 3\\
      input format: string\\
      You can get the road which has the name contains this string

      Query 4\\
      input format: longtitude/latitude/radius\\
      Note that each value should be connected by “/”\\
      You can get a list of POIs details meanwhile you can also have a direct view of their position on the online map.

      Query 5\\
      input format: longtitude/latitude\\
      Note that each value should be connected by “/”\\
      You can get the information of the closest road which is the nearest to your input position.

      Query 6\\
      input format: longtitude/latitude/ longtitude/latitude\\
      Note that each value should be connected by “/”\\
      After submit, we will process the xml file writing in the server and you can get a download link in our website. 
  }%  
}  



\section{Database Design}

\subsection{E-R Model}
\begin{figure}[thpb]
      \centering
      \includegraphics[width=14cm]{ER.pdf}
      \caption{Entity Relationship Diagram}
      \label{fig:ER}
\end{figure}

\subsection{SQL For Table Creation}
\emph{Codes are in `$OSM/TableCreation/create\_tables.py$'}
\begin{verbatim}
CREATE TABLE ways(
          			wayID VARCHAR(12),
                    LineString LINESTRING,
                    name VARCHAR(100), INDEX(name),
                    isRoad VARCHAR(100),
                    otherInfo TEXT,
                    PRIMARY KEY(wayID)
                ) ENGINE=MyISAM
                
CREATE TABLE nodes(
                    nodeID VARCHAR(12),
                    version TINYINT(1), INDEX(version),
                    version BOOLEAN,
                    PRIMARY KEY(nodeID)
                ) ENGINE=MyISAM
                
CREATE TABLE POIs(
                    nodeID VARCHAR(12),
                    position POINT NOT NULL, SPATIAL INDEX(position),
                    planaxy POINT NOT NULL, SPATIAL INDEX(planaxy),
                    name VARCHAR(100), INDEX(name),
                    poitype VARCHAR(100), INDEX(poitype),
                    otherInfo TEXT,
                    PRIMARY KEY(nodeID)
                ) ENGINE=MyISAM
                
create table nonPOIs(
                    nodeID VARCHAR(12),
                    position POINT NOT NULL, SPATIAL INDEX(position),
                    planaxy POINT NOT NULL, SPATIAL INDEX(planaxy),
                    otherInfo TEXT,
                    PRIMARY KEY(nodeID)
                ) ENGINE=MyISAM
                
create table WayNode(
                     wayID VARCHAR(12), INDEX(wayID),
                     nodeID VARCHAR(12), INDEX(nodeID),
                     node_order INT(2),
                     FOREIGN KEY (nodeID) REFERENCES nodes(nodeID),
                     FOREIGN KEY (wayID) REFERENCES ways(wayID)
                ) ENGINE=MyISAM                
\end{verbatim}

\subsection{Data Insertion}
\emph{Codes are in `$OSM/DataInsertion/dumpin.py$'}\\

For the data we parsed from XML, we inserted them into corresponding fields of our created tables.

Notably, if we insert the data directly into the table, the insertion time complexity would be $O(log(N))$, where N is the entries already existed in the table, due to the index (primary key) building process.

Therefore, in order to speed up the insertion process, we disable all the keys before the insertion, and enable them after the insertion. This will ensure every row is inserted in time complexity $O(N)$.

The SQL code is as follows:

\begin{verbatim}
              LOCK TABLE `nodes`, `pois`, `nonpois` WRITE;
              ALTER TABLE `nodes` DISABLE KEYS;
              ALTER TABLE `pois` DISABLE KEYS;
              ALTER TABLE `nonpois` DISABLE KEYS;
              /*...insertion...*/
              ALTER TABLE `nodes` ENABLE KEYS;
              ALTER TABLE `pois` ENABLE KEYS;
              ALTER TABLE `nonpois` ENABLE KEYS;
              UNLOCK TABLES;
\end{verbatim}

The \textbf{LOCK TABLE} is to make sure no other users are writing at the same time.
\subsection{Index}
Besides index for primary keys, we built 8 indexes to accelerate the queries. Especially, in order to speed up the spatial queries, we applied Spatial Index in MySQL. For \textbf{MyISAM} tables, Spatial Index creates an R-tree index. The key idea of the R-tree is to group nearby objects and represent them with their minimum bounding rectangle in the next higher level of the tree. For storage engines that support non-spatial indexing of spatial columns, the engine creates a B-tree index. A B-tree index on spatial values is useful for exact-value lookups, but not for range scans. In our cases, the R-tree is more suitable because required query 4, 5, 6 all include range scans.

\begin{figure}[thpb]
      \centering
      \includegraphics[width=3in]{R-tree.png}
      \caption{R-tree in 2 dimention}
      \label{fig:Rtree}
\end{figure}


\section{Point Mapping}
The longtitude and latitude are used as the absolute coordinates. However, when calculating the distance between two points of given longtitude and latitude, we have to take spherical properties into consideration.

For instance, the distance between $(30.4, 122.1)$ and $(30.4, 122.6)$ is $48.0073$ Km. The distance between $(32.4, 122.1)$ and $(32.4, 122.6)$ is $46.995$ Km. There would be a error about $1$ Km if we ignore the spherical properties of the Earth.

\subsection{Different Ways of Calculating Distances Between Two Given Coordinates}
\begin{enumerate}
\item[a)] Vincenty's formulae are two related iterative methods used in geodesy to calculate the distance between two points on the surface of a spheroid, developed by Thaddeus Vincenty (1975). They are based on the assumption that the figure of the Earth is an oblate spheroid, and hence are more accurate than methods that assume a spherical Earth.
For simplicity and focus on the course related work, here we only provide the link to the Vincenty’s paper without further explanation. 
(\url{http://www.ngs.noaa.gov/PUBS_LIB/inverse.pdf})

\item[b)] Another approach is to map latitude-longitude coordinates to plana coordinates and then calculate the distance in between. We used the definition of Miller’s cylindrical projection, which is more accurate near the equator. Further about the derivation please see the original paper. (\url{http://www.jstor.org/stable/210384})
\end{enumerate}

\subsection{Implementation}
We randomly sampled the start and the destination and got the distribution of the distance error derived using two methods. We concluded that the Vincenty distance, which is more accurate, was 0.66~1.15 times of the Miller distance. This is further explored in our Query 4 and Query 5 design.

\begin{figure}[thpb]
      \centering
      \includegraphics[width=14cm]{Vincenty.png}
      \caption{Relative error between miller distance and vincenty distance}
      \label{fig:Vincenty}
\end{figure}

\section{Solution to Required Queries}
\emph{Codes are in `$OSM/Query/Query.py$'}
\begin{enumerate}
  \item Given a node, return all ways that contain it, and infer whether the node is an intersection of roads, i.e., a crossroad.\\
  \textbf{Solution:} This query is simple. Based on the given node ID, we go to table `waynode' to find the corresponding way ID, and extract information about this way ID in table `ways'.\\
  We have compared the speed of two queries:
  \begin{verbatim}
    SELECT * FROM ways 
        WHERE wayID IN 
        (SELECT wayID FROM waynode WHERE nodeID=givenNodeID);
    or
    SELECT wayID, LineString, name, isRoad, otherInfo 
        FROM ways NATURAL JOIN waynode 
        WHERE waynode.nodeID=givenNodeID;
  \end{verbatim}
  The corresponding runtimes are $0.72 s$ and $8.29 s$. Therefore, we choose to use the first solution instead of the second one. 
  \item Given a way, return all the nodes along the way.\\
  \textbf{Solution:} To realise this query, we designed a solution to lookup the tables multiple times. Firstly, we go to table `waynode' to find the corresponding node ID based on the given way ID. Then find the detail information for these nodes in `POIs' and `Non-POIs'.
  Similar to query 1, we find the solution using the following SQL is faster than using JOIN.
  \begin{verbatim}
    tmpResult$\leftarrow$SELECT * FROM nodes 
                            WHERE nodeID IN 
                            (SELECT nodeID 
                                FROM waynode 
                                WHERE wayID=givenWayID);
    result=[];
    for row in tmpResult:
      if(row['version']==1):
        tmp$\leftarrow$SELECT nodeID, AsText(position) AS position, name, poitype, otherinfo 
                          FROM pois 
                          WHERE nodeID=row['nodeID'];
        result.append(tmp);
      else:
        tmp$\leftarrow$SELECT nodeID, AsText(position) AS position, otherInfo 
                          FROM nonpois 
                          WHERE nodeID=row['nodeID'];
        result.append(tmp);
  \end{verbatim}
  \item Search the name of the road and return information of those matched.\\
  \textbf{Solution:} This query only involve one table `ways'. The SQL is simple as follows:
  \begin{verbatim}
    SELECT * FROM ways WHERE name LIKE ('%input_string%');
  \end{verbatim}
  In order to speed up the query process, we built index for `name' in `ways'.
  \item Query the POIs within a radius of a given location (Longtitude-latitude coordinates).\\
  \textbf{Solution:} In this query, we first convert given coordinate to `planaxy', and draw a circle with a larger radius than required, namely 1.33 times of the original radius. This is to ensure that all nodes within the radius under Vincenty distance will always be included. According to our simulation above. Later we use spatial index to scan for all nodes in a polygon that perfectly circumscribe the desired circle. After we got all possible nodes, we use a linear time examination to find all nodes within the original radius. The core SQL commands responsible for this query are:
  \begin{verbatim}
    SET @poly='Polygon((x-rad, y+rad, 
                        x+rad, y+rad, 
                        x+rad, y-rad, 
                        x-rad, y-rad, 
                        x-rad, y+rad))';
    SELECT nodeID, ST_AsText(position), name, poitype
            FROM POIs
            WHERE MBRContains(ST_GeomFromText(@poly), planaxy);
  \end{verbatim}
  \item Find the closest road to a given GPS coordinate.\\
  \textbf{Solution:} In this query, we first convert given coordinate to `planaxy', and draw a circle with a larger radius than required simiar to Query 4. Next we use a iterative method to find at least on NONPOI point which is closest to the tartget GPS coordinate. To improve efficiency and fast our search, we use an exponentially growing raidus, starting from 10 meters as the initial radius. In the $i^{th}$ attempt, we will search in a circle with radius $10e^{i-1}$ meters. We use the same technique as the above to use the spatial index in a polygon and run filters to get desired answer. The core SQL commands responsible for this query are:
  \begin{verbatim}
    SET @poly=`Polygon((x-rad, y+rad, 
                        x+rad, y+rad, 
                        x+rad, y-rad, 
                        x-rad, y-rad, 
                        x-rad, y+rad))';
    SELECT nodeID, ST_AsText(position)
        FROM nonPOIs
        WHERE MBRContains(ST_GeomFromText(@poly), planaxy);
    SELECT ways.wayid, ways.name, ways.isRoad, ways.otherInfo
        FROM waynode, ways
        WHERE waynode.nodeid=NID and waynode.wayid=ways.wayid and ways.isroad<>'0';
  \end{verbatim}
  \item Implement an API to return the XML in osm format defined in the wiki page, given a rectangular area bounding box (x1, y1, x2, y2) as parameters.\\
  \textbf{Solution:} In this query, we firstly get all the nodes information from `nodes', `pois' and `nonpois'. Then we go to table `waynode' and `ways' to find information of included ways.
  \begin{verbatim}
    SET @poly='Polygon((x1, y1, 
                        x1, y2, 
                        x2, y2, 
                        x2, y1, 
                        x1, y1))';
    SELECT nodeID, AsText(position) as position, name, poitype, otherInfo 
        FROM pois 
        WHERE MBRContains(GeomFromText(@poly), position);
    SELECT nodeID, AsText(position) as position, otherInfo 
        FROM nonpois 
        WHERE MBRContains(GeomFromText(@poly), position);

    SELECT wayID, name, isRoad, otherInfo 
        FROM ways 
        WHERE wayID in 
        (SELECT DISTINCT wayID 
            FROM nonpois 
            NATURAL JOIN waynode 
            WHERE MBRContains(GeomFromText(@poly), position));
    SELECT wayID, name, isRoad, otherInfo
        FROM ways
        WHERE wayID in 
        (SELECT DISTINCT wayID 
            FROM pois 
            NATURAL JOIN waynode 
            WHERE MBRContains(GeomFromText(@poly), position));
  \end{verbatim}
  Then we follow the format defined in the wiki page to output the xml file. In this process, we query the node IDs corresponding a way ID.
  \begin{verbatim}
    SELECT nodeID, node_order FROM waynode
        WHERE wayID=givenWayID order by node_order;
  \end{verbatim}
\end{enumerate}


\section{Interface}
\subsection{Django Framework}
We use Django Framework to design website. Django is an excellent Python Web framework based on MVC model which can be convenient to both query function and interface designing. Here we mainly talk about how our website communicate with query functions in the server and how our website make use of these result to create pretty appearance.
\begin{enumerate}
  \item[a.] We only have one web page. All the actions are performed in one page.

\begin{figure}[thpb]
      \centering
      \includegraphics[width=14cm]{code1.png}
      \caption{Controller}
      \label{fig:Vincenty}
\end{figure}

  One is for our homepage while the other is for our download link which will be included in Download Function
  \item[b.] Basic interaction variable\\
      We mainly have two input variable and two output variable.\\
      The input variable contains query type and query content. Combined these two variables, the server could know which query to choose and execute it with the corresponding input.\\
      The output variable contains query result in string type and query result in number type. The string type result can be used in C part. To pursue a clearer view, we rearrange the result so that their attributes can be stick out a mile. The number type results are mainly made up of coordinates which are used to represent the points and areas on the map plugin.
\end{enumerate}
\newpage
\subsection{Optimized Input Form}
\begin{figure}[thpb]
      \centering
      \includegraphics[width=14cm]{Q2.png}
      \caption{Select menu}
      \label{fig:q2}
\end{figure}

Our input form is characteristic. All the query inputs are entered in the same form. The query type can be selected in the select form. When you select one type, the place holder will also change to remind your input format.

\begin{figure}[thpb]
      \centering
      \includegraphics[width=14cm]{Q3Q4.png}
      \caption{Query3 and Query4 indication}
      \label{fig:q3}
\end{figure}

We use JQuery to manage its action. When user select a query type, it will trigger a series of actions: change the value of query type, change the content of placeholder.
\begin{figure}[thpb]
      \centering
      \includegraphics[width=12cm]{Q5.png}
      \caption{JQuery}
      \label{fig:q5}
\end{figure}
\newpage
\subsection{Download Function}
The query 6 requires us to writing a xml file. In order to present our work in website, we produce download function. The rough idea is to write the xml file and then use wrapper function to send it to the user.

Model part of MVC: it mainly deals with the process of reading and wrapping. It finally returns a download response.
\begin{figure}[thpb]
      \centering
      \includegraphics[width=14cm]{Q6.png}
      \caption{Model of MVC}
      \label{fig:q6}
\end{figure}
Controller part of MVC: it monitors the front operations and when it detects the prescribed link, it will tell the model part to execute the corresponding function.
\begin{figure}[thpb]
      \centering
      \includegraphics[width=14cm]{Q7.png}
      \caption{Controll of MVC}
      \label{fig:q7}
\end{figure}
View part of MVC: a quite simple hyperlink.
\begin{figure}[thpb]
      \centering
      \includegraphics[width=14cm]{Q8.png}
      \caption{View of MVC}
      \label{fig:q8}
\end{figure}
\subsection{Map Plugin}
We use leaflet, an open-source web map plugin in javascipt to produce our map function. The main function is to denote the positions and areas on the map.
We first create a map view centered with the corresponding coordinates. Then we pick out the positions according to the result. In the end, we can’t forget to add the authority so that the map function can be loaded.
\begin{center}
      \includegraphics[width=14cm]{Q9.png}
\end{center}



\section{Division of Work}
\textbf{Zhenfeng Shi:} Schema designing; Database, table, index creation; Required query 1, 2, 3, 6.\\

\textbf{Hongru Zhu:} Algorithms dealing with geo-spatial data designing; Required query 4, 5.\\

\textbf{Chang Zhou:} Demonstration and interfaces.



%% The Appendices part is started with the command \appendix;
%% appendix sections are then done as normal sections
%% \appendix

%% \section{}
%% \label{}

%% References
%%
%% Following citation commands can be used in the body text:
%% Usage of \cite is as follows:
%%   \cite{key}          ==>>  [#]
%%   \cite[chap. 2]{key} ==>>  [#, chap. 2]
%%   \citet{key}         ==>>  Author [#]

%% References with bibTeX database:

\bibliographystyle{model1-num-names}
\bibliography{sample.bib}

%% Authors are advised to submit their bibtex database files. They are
%% requested to list a bibtex style file in the manuscript if they do
%% not want to use model1-num-names.bst.

%% References without bibTeX database:

% \begin{thebibliography}{00}

%% \bibitem must have the following form:
%%   \bibitem{key}...
%%

% \bibitem{}

% \end{thebibliography}


\end{document}

%%
%% End of file `elsarticle-template-1-num.tex'.\documentclass[preprint,12pt]{elsarticle}
